\documentclass[letterpaper,11pt]{article}
\usepackage[round]{natbib}
\usepackage[margin=0.75in,centering]{geometry}
\usepackage{fancyhdr}
\usepackage{amsmath}
\usepackage{amssymb}
\usepackage{graphicx}
\usepackage[pdftex]{hyperref}
\hypersetup{
    pdftitle={On the Rattleback},
    pdfauthor={Dale Lukas Peterson},
    pdfsubject={dynamics and stability of the rattleback},
    pdfkeywords={rattleback, nonholonomic constraints, Kane's method}}

\pagestyle{fancy}
\fancyhead[L]{On the Rattleback, Dale L. Peterson}
\fancyhead[R]{\thepage}  % page number on the right
\fancyfoot[L,C,R]{}  %  No footer on left, center or right, on even or odd pages

\begin{document}
\abstract{This report briefly reviews previous studies of the solid body
  commonly known as a rattleback, celt, or wobblestone.  These solids, whose
  geometric and principal inertia axes are unaligned, exhibit two unexpected
  behaviors: spin reversal, and ``spin bias''.  Spin reversal occurs when the
  body is initially spun about a vertical axis in one direction and after some
  time begins to spin in the opposite direction.  ``Spin bias'' is the behavior
  observed in some rattlebacks which reverse when spun in one direction but not
  the other.  The stability of the equilibrium configuration is examined as a
  function of the spin rate.  For dissipation free models, linear stability
  analysis can only show where spin reversal cannot occur.}

  \section{Brief Literature Review}
  The first published study of the rattleback was performed by G.T. Walker in
  1896 \cite{Walker1896}.  Since that time, the rattleback has been examined by a number of
  authors, both in books \cite{Routh1905a, Bloch2003} and in the flurry of
  papers published in the 1980's by Caughey, Kane, Markeev, Pascal, Bondi,
  Garcia and Hubbard.  With the exception of Garcia and Hubbard, all authors
  assume no-slip of the contact point, and only a few (Kane, Garcia and
  Hubbard) include any dissipation in their models.  Additionally, Garcia and
  Hubbard are the only authors to question the validity of the no-slip model.
  They propose a model which incorporates dry friction and aerodynamic effects,
  compare this model with experimental measurements, and present an explanation
  of the reversal phenomenon based upon the transfer of energy between spinning
  motions and oscillatory motions.

  \section{Model description}
  The shape of the bottom surface of the rattleback is commonly modelled as an
  ellipsoid
  \[
    \left(\frac{x}{a}\right)^2 + \left(\frac{y}{b}\right)^2 + \left(\frac{z}{c}\right)^2 - 1 = 0
  \]
  with semi-diameters of $a, b, c$, or as an elliptic paraboloid (opening in the $-z$ direction)
  \[
    \left(\frac{x}{a}\right)^2 + \left(\frac{y}{b}\right)^2 + z - c = 0
  \]
  with an elliptical cross section and semi-major and semi-minor axes of $a,
  b$, respectively, when $z=c$.  Given the same parameters $a, b$, and $c$, the
  curvatures of these two surfaces are distinct and direct comparisons between
  ellipsoidal and elliptic parabololoid are somewhat difficult.

  The body fixed x-axis is assumed to be the long axis of the rattleback, and
  the body fixed z-axis is assumed to be aligned the vertical when the
  rattleback is in its rest configuration.  All models presented in the
  literature assumed the mass center of the rattleback lies on z axis, but not
  necessarily at the origin of the body fixed coordinate in which the contact
  surface is described (i.e., it can lie above or below the origin).

  The property that gives rattlebacks their interesting behavior is that their
  inertial principal axes are not aligned with the geometric axes; instead, the
  principal axes differ from the geometric axes by a simple rotation about the
  body fixed z-axis.

  Five generalized coordinates completely configure the rattleback with respect
  to an inertial frame.  Two distance coordinates locate the contact point and
  three angle coordinates orient the body fixed axes with respect to the
  inertial axes.  The Euler $ZXY$ (yaw-roll-pitch) angle convention is most
  commonly used.  Of these five coordinates, three are ignorable (cyclic): the
  two distance coordinates and the yaw (heading) coordinate.  Only roll and
  pitch coordinates appear in the dynamic equations.

  If no-slip of the contact point is assumed there are two nonholonomic
  constraints which imply that the rattleback has only three degrees of
  freedom.  If slip is possible (and hence a contact force model is
  prescribed), there are instead five degrees of freedom.

  \section{No-slip, dissipation free equations}
  The equations of motion for ellipsoidal and elliptic paraboloid surfaces have
  been derived with sympy.physics.mechanics and are available for download on
  \href{http://github.com/gilbertgede/pydy_examples}{github.com}.  Without
  dissipation, the model predicts an infinite number of reversals.  This
  property is never observed in real rattlebacks (though some rattlebacks will
  exhibit more than just one reversal).

  The linear dynamics of the reference configuration can be parameterized by
  the spin rate in much the same way the upright linear dynamics of a bicycle
  are parameterized by the forward speed.  Additionally, in the same way that
  the longitudinal dynamics of a bicycle are decoupled (to first order) from
  the lateral dynamics, the spin (yaw) dynamics of a dissipation free
  rattleback are decoupled (to first order) from the roll and pitch dynamics.
  The implications of this, the set of five first order non-linear ODE's, when
  linearized about the reference configuration result in a four dimensional
  linear system, parameterized by the spin rate.

  The stability of the reference configuration indicates that reversal is not
  possible.  Small perturbations in roll and pitch decay exponentially, and
  because the model is the energetically conservative, the spin rate must
  increase a small amount (this behavior will not be captured by the linearized
  model since there are no first order spin dynamics).

  \begin{figure}
    \centering
    \includegraphics[width=0.5\textwidth]{reversals_noslip_nodissipation}
    \caption{No slip, dissipation free rattleback reversals.  Notice the
    difference in period between clockwise reversals and counterclock
  reversals.}
  \end{figure}

  \begin{figure}
    \centering
    \includegraphics[width=0.5\textwidth]{evals_vs_spinrate}
    \caption{No slip, dissipation free rattleback eigenvalues versus spin rate.}
  \end{figure}

  \section{Future work}
  Dynamic equilibria for which roll and pitch are nonzero exist, and their
  stability has yet to be explore in the same fashion as the stability of the
  reference configuration.  In order to explore this, a visualization of the
  equilibria in the (roll, pitch, spin rate) space needs to be generated.  For
  each roll and pitch, there may exist a spin rate (two actually) which
  corresponds to dynamic equilibrium, and the stability of this equilibrium can
  be evaluated.  Plotting level curves of spin rate and regions of stability in
  the (roll, pitch) plane is a good potential candidate for visualizing this.

  Animations of rattleback motion would be a valuable tool in exploring the
  motion of rattlebacks with different parameters, initial conditions, surface
  shapes, and contact models (slip, no-slip).  Work has begun on making these
  animations but is currently incomplete.
% \subsection{No-slip, dissipative equations}
% Adding a simple dissipative torque such as
% \[
%   T = -\sigma \mathbf{\omega}
% \]
% causes only a finite number of reversals to occur.

% \subsection{Dissipative case}
\bibliographystyle{plain}
\bibliography{library}
\end{document}
